%%%%%%%%%%%%%%%%%%%%%%%%%%%%%%%%%%%%%%%%%
% 中文简体模板
% 版本:1.0 (2024.08.21)
% 修改者:ZJW
% 编译器:XeLaTeX
%%%%%%%%%%%%%%%%%%%%%%%%%%%%%%%%%%%%%%%%%

%----------------------------------------------------------------------------------------
%	封面及全局配置
%----------------------------------------------------------------------------------------
\usepackage{fontspec} 
%----------------------------------------------------------------------------------------
%	中文字体 (选择其中1个)
% 
% 	AutoFakeBold=3.5: 粗体的深度
% 	AutoFakeSlant=0.2: 斜体的深度
% 	Path = fonts/: 字体的路径
%----------------------------------------------------------------------------------------
\usepackage{xeCJK}
% \setCJKmainfont[AutoFakeBold=3.5 , AutoFakeSlant=0.2, Path = fonts/]{cwTeX_圓體.ttf}
% \setCJKmainfont[AutoFakeBold=3.5 , AutoFakeSlant=0.2, Path = fonts/]{SIMHEI.ttf} % 黑体
% \setCJKmainfont[
%     Path=fonts/, % 字体文件所在路径
%     BoldFont=MSYHBD.ttc, % 粗体
%     ItalicFont=MSYH.ttc, % 斜体(微软雅黑不区分斜体,所以用常规字体代替)
%     SlantedFont=MSYH.ttc, % 斜体(如果需要,可以设置与常规字体一致)
%     BoldItalicFont=MSYHBD.ttc, % 粗斜体(如果需要,可以设置与粗体一致)
%     AutoFakeBold=3.5, % 自动加粗
%     AutoFakeSlant=0.2 % 自动倾斜
% ]{MSYH.ttc} % 微软雅黑
% \setCJKmainfont[AutoFakeBold=3.5 , AutoFakeSlant=0.2, Path = fonts/]{华文行楷.ttf} 
\setCJKmainfont[AutoFakeBold=3.5 , AutoFakeSlant=0.2, Path = fonts/]{SIMSUN.ttc} % 宋体
% \setCJKmainfont[AutoFakeBold=3.5 , AutoFakeSlant=0.2, Path = fonts/]{SIMKAI.TTF} % 楷体
\XeTeXlinebreaklocale "zh" % 這兩行用來調整中文自動換行
\XeTeXlinebreakskip = 0pt plus 1pt

%----------------------------------------------------------------------------------------
%	英文字体 (选择其中1个)
% 
%----------------------------------------------------------------------------------------
% \usefonttheme{serif} % 使用 serif 字體
\setmainfont[
    Path=fonts/,
    UprightFont=TIMES.TTF,
    BoldFont=TIMESBD.TTF,
    ItalicFont=TIMESI.TTF,
    BoldItalicFont=TIMESBI.TTF
]{Times New Roman} % Times New Roman

\usepackage{color} 
%% 自定义颜色
\definecolor{JLUPurple}{RGB}{102,8,116}
\definecolor{JLUGreen}{RGB}{0,166,82}
\definecolor{JLURed}{RGB}{157,41,51}
\definecolor{JLUBlue}{RGB}{0, 0, 128}

\usepackage{mathtools, amsmath, amsfonts, amsthm, latexsym} 
\usepackage{newtxtext,newtxmath}
\usepackage[justification=centering]{caption} 
\setbeamertemplate{caption}[numbered]
\captionsetup[figure]{font=small, labelfont=md}
\captionsetup[table]{font=small, labelfont=md}
\usepackage{graphicx}  
\usepackage{booktabs}
\usepackage{subfigure} 
\usepackage{multirow} 
\usepackage{array}
\usepackage{enumerate} 
\usepackage{tikz}
\usepackage{natbib}
\usepackage{comment}
\renewcommand{\figurename}{图} 
\renewcommand{\tablename}{表} 

%----------------------------------------------------------------------------------------
%	排版形式 (选择其中1个)
%----------------------------------------------------------------------------------------

\mode<presentation>{
% \usetheme{default} 
% \usetheme{AnnArbor}  
% \usetheme{Antibes}
% \usetheme{Bergen}
% \usetheme{Berkeley}
% \usetheme{Berlin}
% \usetheme{Boadilla}
% \usetheme{CambridgeUS}
% \usetheme{Copenhagen}
\usetheme{Darmstadt}
% \usetheme{Dresden}
%\usetheme{Frankfurt}
%\usetheme{Goettingen}
%\usetheme{Hannover}
%\usetheme{Ilmenau}
%\usetheme{JuanLesPins}
%\usetheme{Luebeck}
% \usetheme{Madrid}
%\usetheme{Malmoe}
%\usetheme{Marburg}
%\usetheme{Montpellier}
%\usetheme{PaloAlto}
%\usetheme{Pittsburgh}
%\usetheme{Rochester}
%\usetheme{Singapore}
%\usetheme{Szeged}
%\usetheme{Warsaw}

%----------------------------------------------------------------------------------------
%	外框形式 (选择其中1个)
%----------------------------------------------------------------------------------------

%\useoutertheme{default}
% \useoutertheme{infolines}
\useoutertheme{miniframes}
% \useoutertheme{smoothbars} 
%\useoutertheme{sidebar}
%\useoutertheme{split}
%\useoutertheme{shadow}
%\useoutertheme{tree}
%\useoutertheme{smoothtree}

%----------------------------------------------------------------------------------------
%	外框的个性化设置
%----------------------------------------------------------------------------------------
%\setbeamercolor{section in head/foot}{fg=black, bg=white} 
%\setbeamertemplate{mini frames}{}  
\setbeamerfont{headline}{size=\scriptsize}
\setbeamerfont{footline}{size=\scriptsize}
\setbeamertemplate{navigation symbols}{} 

%% 方案1:
%\setbeamertemplate{footline} 
%{\leavevmode%
%\hbox{%
%\begin{beamercolorbox}[wd=0.5\paperwidth,ht=3ex,dp=1ex,leftskip=3ex]%
%{author in head/foot}%
%{\scriptsize\textbf{\insertshortauthor}}%
%\end{beamercolorbox}%
%\begin{beamercolorbox}[wd=0.5\paperwidth,ht=3ex,dp=1ex,right]%
%{author in head/foot}%
%\scriptsize \textbf{{\insertframenumber{} / \inserttotalframenumber\hspace*{2ex}}} %頁碼控制選項
%\end{beamercolorbox}%
%}}

%% 方案2
%\setbeamertemplate{footline}[page number] 

%% 方案3:
%\setbeamertemplate{footline}[] 

%----------------------------------------------------------------------------------------
%	顏色主題 (选择其中1个)
%----------------------------------------------------------------------------------------

\usecolortheme{default}
%\usecolortheme{albatross}
%\usecolortheme{beaver}
%\usecolortheme{beetle}
%\usecolortheme{crane}
%\usecolortheme{dolphin}
%\usecolortheme{dove}
%\usecolortheme{fly}
%\usecolortheme{lily}
%\usecolortheme{orchid}
%\usecolortheme{rose}
%\usecolortheme{seagull}
%\usecolortheme{seahorse}
%\usecolortheme{whale}
%\usecolortheme{wolverine}

%----------------------------------------------------------------------------------------
%	顏色主題的个性化设置
%----------------------------------------------------------------------------------------
\setbeamercolor{structure}{fg=JLUPurple} 
%\setbeamercolor{title}{bg=green, fg=black} 
%\setbeamercolor{frametitle}{bg=green,fg=black} 
%\setbeamercolor{normal text}{fg=orange}
%\setbeamercolor{block title}{bg=blue,fg=yellow} 
%\setbeamercolor{block body}{bg=green,fg=red} 
\setbeamercolor{alerted text}{fg=red} 

%----------------------------------------------------------------------------------------
%	enumerate 及 item 的形狀
%----------------------------------------------------------------------------------------

\useinnertheme{rounded} 
%\useinnertheme{circles} 
%\useinnertheme{rectangles}
%\useinnertheme{inmargin} 

%----------------------------------------------------------------------------------------
%	个性化 item 的顏色
%----------------------------------------------------------------------------------------
%\setbeamercolor{item projected}{bg=red}

%----------------------------------------------------------------------------------------
%	个性化页面
%----------------------------------------------------------------------------------------
\setbeamersize{text margin left=0.8cm, text margin right=0.8cm}
\special{papersize=\the\paperwidth,\the\paperheight}
\providecommand{\tabularnewline}{\\}
}

%----------------------------------------------------------------------------------------
%	背景
%----------------------------------------------------------------------------------------
% \setbeamertemplate{background}{\includegraphics[height=\paperheight]{Fig/Background.png}}
\usebackgroundtemplate{%
	\tikz[overlay, remember picture] 
	\node[opacity=0.3, below=-1.25cm, at=(current page.center)] 
	{\includegraphics[scale = 0.14]{Fig/JLU_LOGO.png}}; 
	}
